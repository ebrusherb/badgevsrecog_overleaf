%probably too badge-y
There are many species in which animals have been shown to use badges of status to make inferences about each other. One well known-example is the house sparrow: the size of a male sparrow's black bib is correlated with its physical condition and other sparrows seem to use this information to decide how to interact with a given male \citep{Moller:1987vn,Veiga:1993fk}. Another well-studied example is a species of paper wasp: the extent of black and the number of black patches on a wasp's face are positively correlated with its dominance, suggesting that wasps use these traits when deciding with whom to fight and how aggressively \citep{Tibbetts:2004kx,Tibbetts:2007zr}. Other examples of badges of status in birds include the size of the rusty cap in swamp sparrows, which is correlated with parental investment \citep{Olsen:2010uq}; the bib in great tits \citep{Lemel:1993ve} and the crest in blue tits \citep{Remy:2010fk}, which are used by males to estimate the fighting ability of unknown opponents \citep{Lemel:1993ve}; plumage brightness in house finches, which is correlated with body condition and survival \citep{McGraw:2000qf}; the number of spots in a peacock's tail, which indicates its health \citep{Loyau:2005nx}; the white patch on a collared flycatcher's forehead, which predicts its ability to attain a territory \citep{Part:1997ys}; and pheromones in cockroaches, which affects how other males treat a focal male \citep{Moore:1997kx}. (Badges of status in birds are reviewed in \citep{Jawor:2003bh,Senar:2006dq,Santos:2011ly,Young:2015dq}.) Several factors promote the evolution of a badge that is correlated with quality. Badges are thought to be especially useful in interactions between animals that have no previous information about each other \citep{Lemel:1993ve,Solberg:1997uq,Smith2003AnimalSignals,Remy:2010fk}. For that reason, they may be more likely to evolve if animals do not interact with the same conspecifics many times. In fact, species of birds that spend the winter in fluid flocks are more likely to have badges of status, compared to those that live in stable flocks or occupy stable territories \citep{Rohwer:1975fk,Tibbetts:2009kx}. 

There are also many species that can individually recognize their group mates (reviewed in~\citep{Tibbetts2007IndividualDifferent,Wiley2013SpecificityBehaviour}). In order for individual recognition to evolve, there must be sufficient variation in a group that can be used for recognition and animals must interact frequently enough for there to be an advantage from recognizing conspecifics that have been seen before \citep{Whitfield:1987tg,Sheehan:2014fk}. Species that live in social groups with dominance hierarchies may be more likely to evolve individual recognition because of the importance of keeping track of other animals' positions in the hierarchy \citep{Whitfield:1987tg,Barnard:1979fk}. 

A badge that is informative and ``honest" can evolve and persist if there are costs associated with producing, maintaining, and displaying it. For example, some badges are energetically costly to produce, e.g. \citep{Veiga:1995ys,Buchanan:2001zr,West:2002ly}. Other badges are costly to display when there is a mismatch between the badge and the individual's quality, for example when other individuals punish cheaters ~\citep{Molles:2001kx,Smith2003AnimalSignals,Tibbetts:2004kx} or if dominant animals are more aggressive to strong animals than to weak ones \citep{Rohwer:1981vn,Thompson:2014fk}.   
However, even cost-free signals can maintain their honesty \citep{Dawkins:1991ly,Lachmann:2001uq}.

The correlation between the badge and an underlying trait of interest is only one part of the badge-of-status system. The other part is the learning that animals must engage in in order to make inferences based on the badge. While many studies have addressed the evolution and maintenance of a highly informative badge, few have addressed the evolution of the required type of learning. We therefore focus on the factors that might lead to the evolution of a learning system that relies on badges, rather than the evolution of the badges themselves.

